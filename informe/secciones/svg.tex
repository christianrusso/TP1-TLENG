\section{Geraci\'on del SVG}
Para generar el archivo SVG utilizamos la libreria de python \textbf{svgwrite}\footnote{https://pypi.python.org/pypi/svgwrite}\footnote{Para installar correr: pip install svgwrite}. 
La forma de utilizarla es la siguiente:

\begin{algorithm}[H]
	\# \textit{\footnotesize{Creamos la nueva imagen con los atributos correspondientes}} \\
	img = svgwrite.Drawing(filename = "nombre.svg", atributos) \\
	\# \textit{\footnotesize{Agregamos un atributo Group es decir el tag g}} \\
	group = img.g(atributos) \\
	\# \textit{\footnotesize{Dentro del Group agregamos los tags que querramos, como un Text}} \\
	group.add(img.Text(atributos)) \\
	\# \textit{\footnotesize{Agregamos un tag Line}} \\
	group.add(img.Line(atributos)) \\
	\# \textit{\footnotesize{Se agrega el Group dentro de la imagen}} \\
	img.add(group) \\
	\# \textit{\footnotesize{Se guarda la imagen}} \\
	img.save()
    \caption{Ejemplo de uso de svgwrite}
\end{algorithm}

En particular, en nuestro c\'odigo, la creaci\'on del archivo .svg, la generaci\'on del Group, agregar el Group al svg y guardar el archivo, lo estamos haciendo en \textbf{tp.py}. Luego para cada tag se encargan las propias producciones:  

\begin{itemize}
  \item la producci\'on del operando (OperandNode) escribe en el svg el tag text con los atributos correspondientes
  \item La producci\'on de los parent\'esis (ParenthesisNode) escribe en el svg el tag especial para los parent\'esis
  \item La producci\'on de la division (DivisionNode) agrega el tag Line para la linea devisora
\end{itemize}

De esta forma estamos generando el archivo .svg llamado figura.svg. 

\textbf{Nota:} Con cada ejecuci\'on del c\'odigo se sobreescribe la imagen anterior con la nueva.

