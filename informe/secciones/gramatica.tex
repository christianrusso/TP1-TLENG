\section{Desambiguando la gram\'atica}

Para utilizar un parser LR lo primero que tuvimos que hacer es desambiguar la gram\'atica, dado que nuestra gram\'atica original era:


\begin{center}
\begin{tabular}{ l  l }
  E $\rightarrow$  & E E \\
    & |   E  \^{} E \\
    &  |  E  \_ E \\
    &  |  E \^{} E \_ E  \\
    &  |  E \_ E \^{} E \\
    &  |  E / E \\
    &  |  ( E ) \\
    &  |  \{ E \} \\
    &  |  l \\
\end{tabular}
\end{center}


Y esta \textbf{es ambigua} dado que tenemos mas de un \'arbol de derivaci\'on, por ejemplo para la cadena A \^{} BC \^{} D se la puede derivar de estas maneras:

\begin{center}
\begin{tikzpicture}[level 1/.style={sibling distance=2cm},level 2/.style={sibling distance=2.5cm},level 3/.style={sibling distance=2.5cm},level 4/.style={sibling distance=1.5cm}]
	\node {E}[edge from parent fork down]
		child { node {E}
			child {node{A}}}
		child { node {\^{}}}	
		child { node {E}
			 child {node{E}
			 	child {node{A}}}
			 child {node{\^{}}}
			 child {node{ E}
			 	child {node{E}
					child {node{B}}}
				child {node{E}
				child {node{E}
				child {node{C}}}
				child {node{\^{}}}
				child {node{E}
				child {node{D}}}}}
			}		 		
	;
\end{tikzpicture}



\begin{tikzpicture}[level 1/.style={sibling distance=5cm},level 2/.style={sibling distance=1cm},level 3/.style={sibling distance=0.2cm}]
	\node {E}[edge from parent fork down]
		child { node {E}
			child {node{E}
			child {node{A}}}
			child {node{\^{}}}
			child {node{E}
			child {node{B}}}
			}	
		child { node {E}
			child {node{E}
			child {node{C}}}
			child {node{\^{}}}
			child {node{E}
			child {node{D}}}
			}		 		
	;
\end{tikzpicture}
\end{center}

Entonces para desambiguarla propusimos la siguiente tabla de precedencias y asociaciones.

\begin{center}
  \begin{tabular}{| l | c | r |}
    \hline
    Operaci\'on & Precedencia & Asociatividad \\ \hline
    / & 0 & Izq \\ \hline
    concat & 1 & Izq \\ \hline
    \^{} y\_ & 2& - \\ \hline
    \{ \} y () & 3 & - \\ 
    \hline
  \end{tabular}
\end{center}

Siguiendo esta tabla, obtenemos una gram\'atica \textbf{no ambigua} de la siguiente forma:

\begin{center}
\begin{tabular}{ l  l }
  E $\rightarrow$  & E / C \\
  E $\rightarrow$  &  C \\
  C $\rightarrow$  &  CI \\
  C $\rightarrow$  &  I  \\
  I $\rightarrow$  &  A\^{} A \\
  I $\rightarrow$  &  A \_ A \\
  I $\rightarrow$  &  A\^{} A \_ A \\
  I $\rightarrow$  &  A\_ A \^{} A \\
  I $\rightarrow$  &  A \\
  A $\rightarrow$ &  \{ E \} \\
  A $\rightarrow$ &  ( E ) \\
  A $\rightarrow$ &  l \\
\end{tabular}
\end{center}

Compactada:

\begin{center}
\begin{tabular}{ l  l }
  E $\rightarrow$  & E / C | C\\
  C $\rightarrow$  &  CI | I\\
  I $\rightarrow$  &  A\^{} A |  A \_ A  | A\^{} A \_ A | A\_ A \^{} A | A\\
  A $\rightarrow$ &  \{ E \} | (E) | I\\
\end{tabular}
\end{center}
