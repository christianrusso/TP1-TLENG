\section{Analizador L\'exico}

Para poder generar los Tokens utilizamos la herramienta \textbf{ply}\footnote{http://www.dabeaz.com/ply/}. 

Lo primero que necesitamos fue definir los posibles tokens y sus expresiones regulares.


\begin{center}
\begin{tabular}{ l  l }
  tokens =  & ( SUPERI, \\
    &    SUBI,  \\
    &    DIVIDE, \\
    &    LPAREN,  \\
    &    RPAREN, \\
    &    LBRACE, \\
    &   RBRACE, \\
    &    OPERAND ) \\
\end{tabular}
\end{center}

\begin{center}
\begin{tabular}{ l  l }
  $t\_SUPERI$   &=  \^{} \\
  $t\_SUBI$   &=    $\_$  \\
  $t\_DIVIDE$   &=    $/$ \\
  $t\_LPAREN$    &=    \ (  \\
  $t\_RPAREN$    &=    \ ) \\
  $t\_LBRACE$  &=    \{ \\
  $t\_RBRACE$  &=   \} \\
  $t\_OPERAND$  &=   [\^{} $\setminus$\^{}\_/\{\}$\setminus$($\setminus$)] \\
\end{tabular}
\end{center}


\textbf{Notas}: 
\begin{itemize}
  \item SUPERI y SUBI hacen referencia a super indice y sub indice respectivamente.
  \item Utilizamos t$\_$ como prefijo de las expresiones regulares porque es as\'i como \textbf{ply}\footnote{http://www.dabeaz.com/ply/} las interpreta
  \item La expresi\'on regular t\_OPERAND es negar todos los otros s\'imbolos.
\end{itemize}



Una vez definidas estas dos cosas \textbf{ply}\footnote{http://www.dabeaz.com/ply/} las interpreta y nos genera los tokens correspondientes.

