
\begin{figure}[ptb]
\includegraphics[scale=0.30]{logo.jpg}\hspace{6cm}
\includegraphics[scale=0.90]{logo_dc.jpg}
\end{figure}

%Datos de la caratula
\materia{Teoria de lenguajes}
\titulo{Trabajo pr\'actico }
%\subtitulo{Wiretapping}
\hspace{6cm}

\integrante{Jabalera Gasperi, Fernando}{56/09}{fgasperijabalera@gmail.com}
\integrante{Russo, Christian Sebastián}{679/10}{christian.russo8@gmail.com}
\integrante{Delgado, Alejandro Nahuel}{601/11}{nahueldelgado@gmail.com}

\palabrasClave{TDS, Gram\'atica, Atributos, Ambiguedad}
  % Reconocimiento caras. PCA. Power Method. Deflation. Autovalores. Autovectores. Matriz
  % semi definida positiva.

 \resumen{En el presente trabajo se analiza una gram\'atica de atributos \\
 y se presenta el c\'odigo para poder generar un archivo con extensi\'on svg \\
 que contenga la imagen de una f\'ormula en formato \LaTeX \hspace{0.1cm} ingresada,\\
 partiendo de una gram\'atica ambigua.} %

\hypersetup{%
 % Para que el PDF se abra a página completa.
 pdfstartview= {FitH \hypercalcbp{\paperheight-\topmargin-1in-\headheight}},
 pdfauthor={Delgado, Jabalera, Russo},
 pdfsubject={TP1}
}

\parskip=5pt % 10pt es el tamaño de fuente

% Pongo en 0 la distancia extra entre ítemes.
\let\olditemize\itemize
\def\itemize{\olditemize\itemsep=0pt}

% Acomodo fancyhdr <- Creo que es el encabezado de pagina
\pagestyle{fancy}
\thispagestyle{fancy}
\addtolength{\headheight}{1pt}
\lhead{Delgado, Jabalera, Russo}
\rhead{2$^{do}$ Cuatrimestre 2015}
\cfoot{\thepage}
\renewcommand{\footrulewidth}{0.4pt}




%Pagina de titulo e indice
\thispagestyle{empty}

\maketitle
\tableofcontents

\newpage

